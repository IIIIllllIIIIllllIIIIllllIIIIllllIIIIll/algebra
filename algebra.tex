\listfiles
\documentclass{article}

\usepackage{amsmath}
\usepackage{amssymb}
\usepackage{mathtools}
\usepackage{titlesec}
\usepackage{tikz-cd}
\usepackage[small,nohug,heads=littlevee]{diagrams}
\diagramstyle[labelstyle=\scriptstyle]

\DeclarePairedDelimiter\floor{\lfloor}{\rfloor}
\DeclareMathOperator{\Hom}{Hom}
\DeclareMathOperator{\Set}{Set}
\DeclareMathOperator{\MSet}{MSet}
\DeclareMathOperator{\End}{End}
\def\N{\mathbb{N}}
\def\R{\mathbb{R}}
\def\C{\mathbb{C}}
\def\Z{\mathbb{Z}}

\usepackage[a4paper,margin=1in]{geometry}

\setlength{\parindent}{0cm}
\setlength{\parskip}{1em}

\title{Algebra}
\date{}

\begin{document}
\maketitle

\section*{I. Set Theory and Categories}

\subsection*{1. Sets and Categories}

\subsubsection*{1.1}

Suppose that for any property, we can define a set whose members are items that satisfy that property. Then let $r = \{x : x \not\in x\}$. Then $r \in r \iff r \not\in r$.

In naive set theory and ZFC, this is avoided as the axiom schema of specification (for any property $P$ and set $S$ the set $\{s \in S : s\ \text{satisfies}\ P\}$ exists) requires an existing set $S$. Also, a set of all sets cannot exist, as otherwise we can take $S$ to be such a set.

\subsubsection*{1.2}

Nonempty: any equivalence class $[a]$ contains $a$.

Disjoint: we wish to show that $[a] \ne [b] \implies [a] \cap [b] = \{\}$. Equivalently, $[a] \cap [b] \ne \{\} \implies [a] = [b]$. Suppose $c \in [a] \cap [b]$, and let $a' \in [a]$. Then $a' \sim a \sim c \sim b$, hence $[a] \subseteq [b]$. Similarly $[b] \subseteq [a]$.

Union is $S$: $S$ contains the union as it contains every equivalence class by construction. Let $s \in S$. Then $s \in [s]$, hence the union contains $S$.

\subsubsection*{1.3}

Define $a \sim b$ when there exists $p \in P$ such that $a \in p, b \in p$.

\subsubsection*{1.4}

We count partitions. A partition will have 1, 2 or 3 parts. If it has 1 part it must be the whole set and if it has 3 parts it must be $\{[1], [2], [3]\}$. If it has 2 parts, exactly one part will have 2 items and the other 1 item, and we have 3 choices for the singleton part. Hence there are 5 equivalence relations.

\subsubsection*{1.5}

For $a, b \in \R$ define $a \sim b$ when $|a - b| < 2$. Then $0 \sim 1$ and $1 \sim 2$ but $0 \not\sim 2$.

\subsubsection*{1.6}

Reflexive: for $r \in \R$ we have $r \sim r$ since $r - r = 0 \in \Z$.

Symmetric: suppose $a \sim b$. Then $a - b \in \Z$. Then $b - a = -(b - a) \in \Z$, hence $b \sim a$.

Transitive: suppose $a \sim b, b \sim c$. Then $a - c = (a - b) + (b - c) \in \Z$.

Every class in $\R / \sim$ is represented by a unique real number $r \in [0, 1)$, which is equivalent to all real numbers with that fractional part.

Every class in $\R^2 / \approx$ is represented by a unique pair of real numbers $(r_1, r_2)$ with each $r_i \in [0, 1)$, which is equivalent to all pairs of real numbers whose fractional parts are the representative.

\section*{1. Functions between sets}

\subsubsection*{2.1}

$n!$

\subsubsection*{2.2}

Let $f: A \to B$. We wish to show that $f$ has a right-inverse iff it is surjective.

$\implies$: let $b \in B$. Suppose the right inverse is $g$; then $f \circ g = id_B$ hence $b = f(g(b)) \in f(A)$.

$\impliedby$: we define $g: B \to A$ as follows. Let $b \in B$. Choose $a \in f^{-1}(b)$, which is nonempty since $f$ is surjective, then set $g(b) = a$. Then $g$ is a right inverse. Proof: for all $b \in B$ we have $f(g(b)) = f(a) = b$.

\subsubsection*{2.3}

Let $f: A \to B$ be a bijection and $f^{-1}$ its inverse. Since $f^{-1}$ is a right inverse of $f$, we have $f \circ f^{-1} = id_B$ hence $f$ is a left inverse of $f^{-1}$. By a similar argument, it is a right inverse as well; hence $f^{-1}$ is a bijection.

Let $g: A \to B, f: B \to C$ be bijections. Then $(f \circ g) \circ (g^{-1} \circ f^{-1}) = id_C$, hence $f \circ g$ is surjective. By a similar argument, it is injective.

\subsubsection*{2.4}

Reflexive: for any set $A, |A| = |A|$ since $id_A$ is a bijection.

Symmetric: suppose $|A| = |B|$, that is, there is a bijection $f: A \to B$. Then $f^{-1}: B \to A$ is a bijection, hence $|B| = |A|$.

Transitive: suppose $|A| = |B|, |B| = |C|$ with bijections $f$ and $g$. Then $f \circ g$ is a bijection between $A$ and $C$.

\subsubsection*{2.5}

A function $f: A \to B$ is an epimorphism (is epic) if for all sets $Z$ and all functions $\alpha', \alpha'': B \to Z$ we have $\alpha' \circ f = \alpha'' \circ f \implies \alpha' = \alpha$. A function is surjective iff it is epic.

$\implies$: Let $f: A \to B$ be surjective and $\alpha', \alpha'': B \to Z$ such that $\alpha' \circ f = \alpha'' \circ f$. Then $f$ has a right inverse $g$ and $\alpha' \circ f \circ g = \alpha'' \circ f \circ g$. Hence $\alpha' = \alpha''$.

$\impliedby$: Let $f: A \to B$ be epic and let $b \in B$. We wish to show that $b \in f(A)$. Let $Z = \{0, 1\}$ and suppose $\alpha'$ and $\alpha''$ disagree only on $b$, that is, $\alpha'(c) = \alpha''(c) \iff c \ne b$. Then $\alpha' \ne \alpha''$. Suppose $b \not\in f(A)$. Then for all $a \in A \alpha' \circ f(a) = \alpha'' \circ f(a)$ (proof: $f(a) \ne b$, hence $\alpha'(f(a)) = \alpha''(f(a))$), which contradicts $f$ is an epimorphism.

\subsubsection*{2.6}

For any $f: A \to B$ define $\phi_f: A \to A \times B$ by $\phi_f(a) = (a, f(a))$. Then $\phi_f$ is a section (right inverse) of $\pi_A$. Proof: for all $a \in A$ we have $\pi_A \circ \phi_f (a) = \pi_A (a, f(a)) = a$. In fact every section of $\pi_A$ corresponds to a unique function, since the $\pi_B(\pi_A^{-1}(x)) = B$.

\subsubsection*{2.7}

For all $f: A \to B$ let $\phi_f: A \to \Gamma_f$ be given by $\phi(a) = (a, f(a))$. Then $\pi_A$ is a left inverse of $\phi_f$ by exercise 2.6, hence $\phi_f$ is injective. Also $\phi_f$ is surjective. Proof: suppose $y = (a, b) \in \Gamma_f$. Then $b = f(a)$, hence $y = \phi_f(a)$.

\subsubsection*{2.8}

The equivalence relation is: $r_1 \sim r_2 \iff e^{2\pi ir_1} = e^{2\pi ir_2} \iff r_1 - r_2 \in \Z$. Hence the surjection maps $r$ to (the equivalence class represented by) its fractional part; $\bar f$ maps $[r]$ to $e^{2\pi ir}$, with the unit circle as codomain; the injection is the inclusion map (of the unit circle in the complex plane).

\subsubsection*{2.9}

Let the bijection between $A' and A''$ be $f$ and the bijection between $B'$ and $B''$ be $g$. Define a bijection $h: A' \cup B' \to A'' \cup B''$ as follows: $h(x)$ is $f(x)$ if $x \in A'$ otherwise $g(x)$. This is well-defined as exactly one of $x \in A', x \in B'$ is true. The inverse can be defined explicitly similarly.

To form the disjoint union of $A$ and $B$ be produce disjoint copies; suppose we do this in one way to get $A'$ and $B'$, and another way to get $A''$ and $B''$. Since they are copies, $A' \cong A'', B' \cong B''$, hence the disjoint unions are isomorphic (as sets).

\subsubsection*{2.10}

Any function from $A$ to $B$ is defined by the value of $f(a)$ for all $a \in A$; the choice for each $a$ is independent, there are $|A|$ choices and for each choice we may choose $|B|$ ways, so there are $|A| \times |A| \times \ldots |A|$ different functions.

For an explicit bijection, we can use the fact that any finite set can be well-ordered to order the elements of $|B^A|$ lexicographically on their valuations $(f(a_1), f(a_2) \ldots )$.

\subsubsection*{2.11}

A subset $S \subseteq A$ is determined uniquely by its indicator function $f_S \in 2^A$ given by $f_S(a) = 0 \iff a \in S$.

\section*{3. Categories}

\subsubsection*{3.1}

We compose two morphisms in $C^{op}$ by composing the two underlying morphisms in $C$ ``the other way". That is, suppose $f \in \Hom_{C^{op}}(A, B) = \Hom_C(B, A)$ and $g \in \Hom_{C^{op}}(B, C) = \Hom_C(C, B)$. Let composition in $C$ be denoted by $\cdot$ and composition in $C^{op}$ by $\circ$. Then define their composition $g \circ f = f \cdot g$.

1. The identity homorphism exists because $1_A \in \Hom_C(A, A) = \Hom_{C^{op}}(A, A)$.

2. We explicitly constructed the composition in $C^{op}$, hence it exists.

3. $(f \circ g) \circ h = h \cdot (g \cdot f) = (h \cdot g) \cdot f = f \circ (g \circ h)$.

4. $f \circ 1_A = 1_A \cdot f = f$; similar proof for left inverse.

\subsubsection*{3.2}

$|\End(A)| = |\Hom(A, A)| = |A^A| = |A|^{|A|}$.

\subsubsection*{3.3}

That means it is a left and right identity. Left identity means for all morphisms $f: a \to b, 1_bf = f$. Proof that $1_b$ is a left identity: let $f: a \to b$. By our definition of composition, $1_bf = (a, b)$ (let $c = b$ on p21), which is identically $f$. The proof for right identity is similar.

\subsubsection*{3.4}

No, since $n \not< n$, so $\Hom(n, n)$ is empty.

\subsubsection*{3.5}

$\subseteq$ in 3.4 is the $\sim$ relation in 3.3 (that is required to be reflexive and transitive).

\subsubsection*{3.6}

1. Let the identity in $V$ be the identity matrix.

2. Let $f: x \to y, g: y \to z$. The composition $gy$ is exactly the matrix product $gy$, which is the correct size ($gy: x \to z$ is a $z \times y$ matrix).

3. Associativity follows from associativity of matrix multiplication.

4. The identity matrix is an identity with respect to matrix multiplication.

An object $n$ of this category can be thought of as $\R^n$, and the morphisms $\Hom(m, n)$ as linear functions $\R^m \to \R^n$. (Eg use the standard basis on $\R^m$ and $\R^n$ to convert a linear function to an appropriately sized matrix).

\subsubsection*{3.7}

Let us denote this category as $C^A$. Then the objects of $C^A$ are morphisms of $C$. Given two objects in $f, g \in Obj(C^A)$, suppose $f \in \Hom_C(A, Z_1)$ and $g \in \Hom_C(A, Z_2)$. Define a $C^A$-morphism from $f$ to $g$ to be a $C$-morphism $\sigma$ such that the diagram commutes.

\begin{tikzcd}
 & A \arrow[ldd, "f"'] \arrow[rdd, "g"] &  \\
 &  &  \\
Z_1 \arrow[rr, "\sigma"] &  & Z_2
\end{tikzcd}

1. An identity $C^A$-morphism on the $C^A$-object $f$ is a $C$-morphism $h$ such that

\begin{verbatim}
A --f--> Z
|       /
|      /
f     h
|    /
|   /
V  V
Z
\end{verbatim}

commutes; $1_Z$ does the job.

2. Suppose $f, g, h$ are $C^A$-objects and $p$ is a $C^A$-morphism from $f$ to $g$ and $q$ is a $C^A$-morphism from $g$ to $h$. Then the following diagram commutes.

\begin{verbatim}
Z3 <--h--A --f--> Z1
  ^      |       /
   \     |      /
    q    g     p
     \   |    /
      \  |   /
       \ V  V
         Z2
\end{verbatim}

By removing the vertical line and composing $p$ and $q$,

\begin{verbatim}
Z3 <--h--A --f--> Z1
  ^              /
   \            /
    \          /
     \        /
      \      /
       --qp--
\end{verbatim}

commutes; hence define composition in $C^A$ as composition in $C$.

3. Associativity follows from associativity in $C$.

We will also verify the claim in example 3.8.

\begin{verbatim}
{*} --f--> S
 |       /
 |      /
 g     h
 |    /
 |   /
 V  V
  T
\end{verbatim}

The requirement is that $h$ commutes, that is, considered as set-functions, $g = h \circ f$. Since $g$ is completely determined by its value on $*$, this means $g(*) = h(f(*))$, or $t = h(s)$.

\subsubsection*{3.8}

The objects of $C$ are infinite sets and the morphisms in $C$ are set-functions between them (that is we define $\Hom_C(A, B) = \Hom_{Set}(A, B)$, forcing $C$ to be a full subcategory of $Set$). Identity, composition and associativity all follow from $\Set$.

\subsubsection*{3.9}

WIP

A function maps between two sets; we must generalize this concept to allow for maps between two multisets. How we do this determines the structure of morphisms in $\MSet$. (In all cases, though, members of $Obj(\MSet)$ will be pairs $(S, \sim)$ where $\sim$ is an equivalence relation on $S$).

1. We can treat $(S, \sim)$ as a partitioned set; then mfunctions between them will be functions between partitions.

2. We can simply inherit the morphisms from $S$, allowing morphisms to ignore $\sim$ altogether. The downside is that mfunctions are now sensitive to which representative of a given element you feed it.

3. We can generalize functions to "multivalued functions" (eg the $z \to z^{1/3}$ operator on $\C$); TBD

4. We can generalize functions to "regular functions whose image contain elements with multiplicities"; for finite multiplicities this is like a regular function $x \to (y, n)$ where $x, y$ are our conceptual domain and codomain and $n$ is the multiplicity. A contrived example is: given a fixed function $f$ over $\R$, we could map $x$ to $(y, n)$ where $y = f(x)$ and $n$ is the algebraic multiplicity of the root of the function $f'(t) = f(t) - f(x)$ at $y$ (hence $n$ would mostly be 1, and at stationary points it would increase).

\subsubsection*{3.10}

The object $\Omega = \{0, 1\}$ will do. A subobject (subset) $A'$ of an object (set) $A$ can be identified with a morphism (set-function) $f: A \to \Omega$ by setting $a \in A' \iff f(a) = 1$.

\subsubsection*{3.11 - $C^{A,B}$}

A $C^{A,B}$-object is a pair of $C$-morphisms $(f, g)$. A $C^{A,B}$-morphism between $(f_1, g_1)$ and $(f_2, g_2)$ is a $C$-morphisms $\sigma$ such that $f_2 = \sigma f_1$ and $g_2 = \sigma g_1$. Composition is inherited from $C$.

\begin{tikzcd}
 &  & A \arrow[lldd, "f_1"'] \arrow[dd, "f_2" description] \arrow[rrdd, "f_3"] &  &  \\
 &  &  &  &  \\
Z_1 \arrow[rr, "\sigma"] &  & Z_2 \arrow[rr, "\tau"] &  & Z_3 \\
 &  &  &  &  \\
 &  & B \arrow[lluu, "g_1"] \arrow[uu, "g_2" description] \arrow[rruu, "g_3"'] &  &
\end{tikzcd}

\subsubsection*{3.11 - $C^{\alpha, \beta}$}

Let $\alpha: C \to A$ and $\beta: C \to B$. A $C^{\alpha, \beta}$-object is a pair of $C$-morphisms $(f, g)$ with a common target $Z$ such that $f\alpha = g\beta$. A $C^{\alpha, \beta}$-morphism between objects $(f_1, g_1)$ with common target $Z_1$ and $(f_2, g_2)$ with common target $Z_2$ is a $C$-morphism $\sigma: Z_1 \to Z_2$ such that $f_2 = \sigma f_1$ and $g_2 = \sigma g_1$.

\begin{tikzcd}
 &  &  & A \arrow[ld, "f_1"] \arrow[lld, "f_2", bend right] \arrow[llld, "f_3"', bend right] &  \\
Z_3 & Z_2 \arrow[l, "\tau"] & Z_1 \arrow[l, "\sigma"] &  & C \arrow[lu, "\alpha"'] \arrow[ld, "\beta"] \\
 &  &  & B \arrow[lu, "g_1"'] \arrow[llu, "g_2"', bend left] \arrow[lllu, "g_3", bend left] &
\end{tikzcd}

\subsection*{4. Morphisms}

\subsubsection*{4.1}

Proof ommitted.

\subsubsection*{4.2}

The are relations which are also symmetric. Proof: let $C$ be a category of this kind and let $A \sim B$, that is, $f \in \Hom_C(A, B)$. By the groupoid property there exists a left inverse $g \in \Hom_C(B, A)$, hence $\Hom_C(B, A)$ is nonempty, hence $B \sim A$.

\subsubsection*{4.3}

Let $\beta' \circ f  = \beta'' \circ f$. Let $g$ be a left inverse of $f$. Then $\beta' \circ f \circ g = \beta'' \circ f \circ g$, hence $\beta' = \beta''$.

TBD

\subsubsection*{4.4}

As long as $Obj(C)$ is nonempty, the structure $C_{nonmono}$ constructed as such cannot work since the identity function is a monomorphism, hence not a homomorphism in $C_{nonmono}$.

\subsubsection*{4.5}

TBD

\subsection*{5. Universal Properties}

\subsubsection*{5.1}

Let $Z$ be final in $C$. Consider $Z$ as an element of $C^{op}$ and let $Z'$ be any element of $C^{op}$. There is a unique $C$-morphism between $Z$ and $Z'$, hence there is a unique $C^{op}$-morphism between $Z'$ and $Z$.

\subsubsection*{5.2}

$0$ is inital since for all sets $S$ there is a unique function from $0$ to $S$, the empty function. Let $T \ne 0$ be a set. Then $T$ must have at least one element, say $t \in T$. Consider $Hom(T, P)$ where $P = \{a, b\}$ where $a = \{\}, b = \{a\}$ is a set with two elements. If $Hom(T, P)$ is nonempty, say $f \in Hom(T, P)$, then either $f(t) = a$ or $f(t) = b$; either way we can construct $f'$ which differs from $f$ in its value of $t$.

\subsubsection*{5.3}

Let $F$ be final. We show that there is a unique automorphism. This follows because $F$ is final and an automorphism is $F \to F$.

Let $F_1, F_2$ be final. We show that there is a unique isomorphism between $F_1$ and $F_2$. Since $F_1$ is final there is a unique morphism $f : F_2 \to F_1$. It suffices to show that $f$ is an isomorphism. Since $F_2$ is final there is a unique morphism $g : F_1 \to F_2$. Then $gf = 1_{F_2}$ and $fg = 1_{F_1}$.

\subsubsection*{5.4}

Let $(I, i)$ be an initial member of $Set^*$. Then there must be exactly one morphism from $(I, i)$ to any other object $(S, s)$. All singleton sets (that is, $(\{i\}, i)$) are initial, since given any $(S, s)$ the unique morphism maps $i$ to $s$. All non-singleton sets are not initial since there are at least two morphisms to $(S, s)$ for $|S| > 1$.

Let $(F, f)$ be a final member of $Set^*$. Then there must be exactly one morphism from any other object $(S, s)$ to $(I, i)$. All singleton sets are final with the morphism mapping $s$ to $i$. All non-singleton sets are not final (same argument as above).

\subsubsection*{5.5}

We define the category $C$ explicitly. A $C$-object is a Set-morphism $\sigma: A \to Z$ such that $a \sim a' \implies \sigma(a) \sim \sigma(a')$, and $C$-morphisms between $\sigma_1$ and $\sigma_2$ are Set-morphisms $f$ such that $\sigma_2 = f \sigma_1$. The final objects in this category are the Set-morphisms whose target are singleton sets.

\subsubsection*{5.6}

Let $m, n$ be objects of this category and let $f$ be a final objects in $C_{m, n}$. Then $f | m$ and $f | n$. Furthermore, for all $f'$ such that $f' | m$ and $f' | n$ we must have $f' | f$. Hence $f$ is the gcd of $m$ and $n$.

Let $f$ be an initial object in $C^{m, n}$. Then $m | f$ and $n | f$. Furthermore, for all $f'$ such that $m | f'$ and $n | f'$, we have $f | f'$. Hence $f$ is the lcm of $m$ and $n$.

\subsubsection*

The definition of $A \sqcup B$ given makes it a coproduct in $Set$.


\end{document}
